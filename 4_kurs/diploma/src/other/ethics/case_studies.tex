\documentclass[a4paper,12pt]{article}

\usepackage[utf8]{inputenc}
\usepackage[T2A]{fontenc}

\usepackage[english, ukrainian]{babel}

\usepackage{amsmath,amssymb,mathrsfs}

\usepackage{amsthm}
\usepackage[mathscr]{eucal}
%\usepackage[dvips]{graphicx}
\usepackage[pdftex]{graphicx}
\usepackage{epstopdf}


%\usepackage[centerlast,small]{caption2}
\usepackage{enumerate}
\usepackage{verbatim}
\usepackage{layout}
\usepackage{cite}
%%% remove comment delimiter ('%') and specify parameters if required
%\usepackage[dvips]{graphics}

\usepackage{hyperref}
\tolerance=1000
\hypersetup{colorlinks=true}

\pagestyle{empty}

\usepackage{geometry} % Можливість задавати поля
\geometry{top=15mm}
\geometry{bottom=15mm}
\geometry{left=15mm}
\geometry{right=10mm}

\usepackage{extsizes} % Можливість зробити 14-й шрифт(8pt, 9pt, 10pt, 11pt, 12pt, 14pt, 17pt, 20pt.)



\begin{document}


%%% remove comment delimiter ('%') and select language if required
%\selectlanguage{spanish}

    \begin{center}
        \textbf{Робота \\}
        \textit{
            студента 1 курсу денної форми навчання \\
            факультету комп'ютерних наук та кібернетики \\
            Маматюсупова Дмитрія Олеговича \\
            на тему \\
            Ситуаційний аналіз (Case-studies)  на прикладі  професійної ситуації, \\
            що містить дилему, з позиції відповідності \\
            професійним морально-етичним принципам\\}

    \end{center}

    \section{Опис ситуації}
    Юлі всього 5 років. Але батьки були вимушені вже два рази переводити її в інший дитячий садок,
    тому що її присутність у групах, за словом вихователів, небезпечна для здоров'я оточуючих.
    Юля то носиться в групі, як вихр, то бросає у дітей іграшки, коли вони говорять, що-небудь неприємне для неї,
    то намагається вихватити у вихователя тарілку з супом, проливши її вміст на дітей. В тихий час вона не спить,
    а гучно співає. Поява Юлі в новому дитячому саду в перший же день викликала занепокоєння батьків
    інших дітей. Вихователі звернулись до психолога дитсадка. Виявлось, що вдома з
    її поведінкою проблем не має. Юля слухняна і некапризна: папа тримає
    її в його їжових рукавацях. Вихователі намагалися використовувати в суспільстві з
    Юлією лише демократичний стиль впливу: довго поясняли їй, чому треба вчиняти тим чи іншим чином,
    намагаючись робити зауваження в м'якій формі. Такий підхід був протипоставлений домашньому вихованню,
    де за будь-яку дію, дівчаток супроводжувалось окриком, зауваження: «Відійди!», «Перестань!».
    З нею часто вчиняли фізичні розправи. Ці підходи вступають в противореччя один з іншим,
    а непослідуваність дорослих порождала всі нові та нові капризи дитини.

    \section{Виявлення моральної дилеми . (Виділіть цінності / пріоритети всіх зацікавлених сторін, що включені в проблему.}

    Батьки - Гроші - Суспільство - Вихователь - Дитина.


    Цінності вихователя визначити неможливо. Якщо він є професіоналом то він не має апелювати к емоціям в процесі виховання.
    Вихователь є непрямою причиною системи взаємовідносин у колективі. Коли він не знає виходу він йде за допомогою.

    Батьки. Знову їх поведінка незрозуміла і маловірогідно, що є наслідком якихось правил або принципів.

    Дитина. А де в ситуації формулювання проблем дитини? Дитина розповсюдила свої проблеми на оточуючих. Немов щось погане.


    \section{Перерахуйте основні причини відношення до проблеми/дилеми з етичної точки зору. (Хто несе найбільшу моральну /матеріальну шкоду внаслідок проблеми)}
    Інтегральна матеріальна шкода по спаданню: дитина -> батьки -> вихователь -> психолог
    Моральна шкода: дитина -> вихователь -> батьки -> психолог

    \section{Визначте пріоритети в діяльності всіх учасників конфлікту /дилеми. Яку етичну оцінку їм можна дати?}
    Логіка.


    Парадигма. Всі розв'язують задачі максимально оптимально.


    Батьки. Не хочуть займатися вихованням дитини.

    Вихователь. Хоче професійно вчиняти.

    Дитина. - . - (сильно залежить від інтерпретації, наприклад хоче бути почутою(чомусь))

    Психолог. Той чия присутність дає вихователю не бути господарем над дітьми(силою взаємодії з батьками). Так вихователь краще працює.

    Батьки інших дітей. Захищають своїх дітей і непрямо себе доступними їм способами внесенням противіччя вихователі не виховають в загал.

    \section{Якщо присутня ситуація, що про проблему знали керівники, але не зробили нічого аби завадити виникненню даної проблеми/дилеми/конфлікту. Якими чинниками, на ваш погляд вони керувалися?}

    Інтерпретую так. Чому при вирішенні проблеми ніхто нічого не спитав у дитини? Тому що всім чхати.
    І взагалі співвідношенню дорослий > дитина складно бути оспореним.(Дорослий поясняє дитині, і тд)

    \section{Які причини бездіяльності по усуненню проблеми зі сторони владних структур? Чи можна це назвати простою халатністю?}
    `Владні структури детостатньо владні.`

    Ситуація взагалі неоднозначна. Є наявність конфілку з центром у дитині. Чи добре це для дитини, не впевнений, бо
    якщо банально звикнути до правила `я в центі конфілкту` може бути складно в майбутньому.
    Органи виховання не контролюють такі маленькі конфлікти. Взагалі можливо можна знайти `підхід' до багатьох дітей.
    Чому це не було зроблено одразу? Можливо через правило не лізь поперед батька в пекло. Змінювали дитсадок(ідея вірна).
    Ну а батьки тільки згодом почали думати про дитину. Я взагалі не педагог і дидиктиці не навчався в Гарварді, що вам треба)

    `Но осадок остался`(в дитини)

    \section{Показати можливі шляхи розв’язання, застосовуючи моральну аргументацію з позиції держави/організації/ ін.установ за необхідності/ваша власна позиція}
    Слухати один одного максимально, зрозуміти хто чого хоче, не створювати зайвих конфліктів, прийти до компромісу, інакше змінити контекст(дитсадок, систему виховання, місце проживання)
    Також батькам слід більше самим займатись вихованням і зрозуміти, що в вихованні їх дітей вони більше всіх зацікавлені(окрім дитини звісно).

    \pagestyle{empty}


\end{document}
