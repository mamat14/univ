\documentclass[a4paper,12pt]{article}

\usepackage[utf8]{inputenc}
\usepackage[T2A]{fontenc}

\usepackage[english, ukrainian]{babel}

\usepackage{amsmath,amssymb,mathrsfs}

\usepackage{amsthm}
\usepackage[mathscr]{eucal}
%\usepackage[dvips]{graphicx}
\usepackage[pdftex]{graphicx}
\usepackage{epstopdf}


%\usepackage[centerlast,small]{caption2}
\usepackage{enumerate}
\usepackage{verbatim}
\usepackage{layout}
\usepackage{cite}
%%% remove comment delimiter ('%') and specify parameters if required
%\usepackage[dvips]{graphics}

\usepackage{hyperref}
\tolerance=1000
\hypersetup{colorlinks=true}

\pagestyle{empty}

\usepackage{geometry} % Можливість задавати поля
\geometry{top=15mm}
\geometry{bottom=15mm}
\geometry{left=15mm}
\geometry{right=10mm}

\usepackage{extsizes} % Можливість зробити 14-й шрифт(8pt, 9pt, 10pt, 11pt, 12pt, 14pt, 17pt, 20pt.)



\begin{document}


%%% remove comment delimiter ('%') and select language if required
%\selectlanguage{spanish}

    \begin{center}
        \textbf{Рецензія \\}
        \textit{
            студента 1 курсу денної форми навчання \\
            факультету комп'ютерних наук та кібернетики \\
            Маматюсупова Дмитрія Олеговича \\
            на сериал \\
            Містер Робот}

    \end{center}

    \section{Попередньо}
    \url{https://en.wikipedia.org/wiki/Mr.Robot}


    \url{https://en.wikipedia.org/wiki/Film_criticism}

    In general, film criticism can be divided into two categories: journalistic criticism which appears regularly in newspapers,
    magazines and other popular mass-media outlets; and academic criticism by film scholars who are informed by film
    theory and are published in academic journals. Academic film criticism rarely takes the form of a review;
    instead it is more likely to analyse the film and its place within the history of its genre, or the whole of film history.

    Також я подивився, що люди говорять, щоб зрозуміти ставлення інших

    \url{https://youtu.be/6PMQvjwCdEM}(мр Робот про любов та особистість)

    \url{https://www.youtube.com/watch?v=SSEVmgebSKw}(мр Робот про бога)

    \url{https://www.youtube.com/watch?v=4yKsIdr_PNU}(інтервью з Еліотом з мантрою F@ck The Society)


    \section{Відгук}
    `А це точно анонімний відгук?'

    Відгук засновуючись на теорії филімів я не зроблю.
    Фільм не є проривом, кібератаки вже давно відомо, психичні розлади хоч у дуже популярному Бійцівскький клуб(1999).

    Коли Елліот Олдерсон - головний герой «Містера Робот », якого грає Рамі Малек - вводить код у свій термінал Linux,
    це все фактично програмування.
    Файли, звичайно, винайдені, і деякі особливо блискучі хакі Еліота - це програми, які були написані раніше,
    так що їх не потрібно спостерігати на екрані.
    Але в інший час командні рядки та обхідні шляхи програмування є дуже реалістичними.
    Елліот потрапляє на сервери і судо наказує їм відмовитися від своїх секретів; коли він розмовляє про шифрування,
    він використовує ті самі терміни, що і ваш Wi-Fi маршрутизатор.
    Його люди зустрічаються на каналах IRC. Кожен заголовок епізоду насправді є ім'ям файлу - звісно,
    що закінчується на ".avi" - із розширення, подібним до незаконно(умовно) розподілених торрент файлів.
    І природно, перший епізод - це не перший епізод, а нульовий; програмісти починають рахувати з нуля(а ще деякі математики).

    Фильм зосереджений на інтернет безпеці і судячи з відгуків кібератаки змодельовані гарно.
    Важливою частиною відгуку є те, що чим більше думаєш про цей фільм тим менше хочеться писати. (по об'єму)
    Взагалі серіал знятий більш-менш якісно, що є плюсом. Коли я його дивився то не зрозумів аудиторії.
    Теорія заговору теж зовсім ідея не нова. Як мінімум дуже популярні теорії датуються щонайменш 2001 роком.
    Психотерапія і наркотики - теж тема відома. Здавалось нічим цей серіал уваги не заслуговує.
    Втім його подивилось багато людей. Я думаю це через `ін'єкції` в фільм деяких тригерів, які дають такий загал.
    Бо кому взагалі цікаво дивитись фільм про хакерів? Хакерам? Але фільм не тільки про хакерів.

    \section{Висновок}
    Фільм знятий нормально(як на мене), але не є дуже цікавим бо неясно в такому великому обсязі охоплених тематик
    вловити які-небудь конкретні ідеї.


\end{document}
