%! Author = dmytriim
%! Date = 20.03.2021

\documentclass[a4paper,12pt, titlepage]{article}

\usepackage[utf8]{inputenc}
\usepackage[T2A]{fontenc}

\usepackage[english, ukrainian]{babel}

\usepackage{amsmath,amssymb,mathrsfs}
\usepackage{amsthm}
\usepackage[mathscr]{eucal}
%\usepackage[dvips]{graphicx}
\usepackage[pdftex]{graphicx}
\usepackage{epstopdf}
\usepackage{epigraph}

%\usepackage[centerlast,small]{caption2}
\usepackage{enumerate}
\usepackage{verbatim}
\usepackage{layout}
\usepackage{cite}
%%% remove comment delimiter ('%') and specify parameters if required
%\usepackage[dvips]{graphics}

\usepackage{hyperref}
\tolerance=1000
\hypersetup{colorlinks=true}

\pagestyle{empty}

\usepackage{geometry} % Можливість задавати поля
\geometry{top=15mm}
\geometry{bottom=15mm}
\geometry{left=15mm}
\geometry{right=10mm}

\usepackage{extsizes} % Можливість зробити 14-й шрифт(8pt, 9pt, 10pt, 11pt, 12pt, 14pt, 17pt, 20pt.)

\title{\begin{otherlanguage*}{ukrainian}
           Контрольна робота
\end{otherlanguage*}}
\author{\begin{otherlanguage*}{ukrainian} Маматюсупов Дмитрій \end{otherlanguage*}}
\date{\begin{otherlanguage*}{ukrainian} Листопад 2020 \end{otherlanguage*}}

\begin{document}
    \maketitle

    \tableofcontents

    \section*{Завдання 1}
    \subsection*{Властивості основної пентади програмування}
    Пентада основних понять програмування
    Вище було побудовано три тріади розвитку, що в
    сукупності вводять п’ять основних понять
    програмування (на професійному рівні).
    Які?
    Ці п’ять понять та їх співвідношення будемо називати
    пентадою основних понять програмування. Ця пентада
    утворена трьома обертами розвитку (конкретизації). Її
    розвиток почався з філософського рівня.

    Пентада основних понять програмування виділяє
    основні аспекти понять програмування. Так, для
    поняття програми основними аспектами будуть:
    \begin{enumerate}
        \item адекватність;
        \item прагматичність;
        \item обчислюваність;
        \item генетичність.
    \end{enumerate}
    Для процесу програмування основні аспекти це
    \begin{enumerate}
        \item проблемна орієнтованість;
        \item експлікативність.
    \end{enumerate}

    \begin{enumerate}
        \item Df 1: програмування є процес побудови програм. Це
        визначення поєднує лише два поняття пентади та
        абстрагується від інших понять.
        \item Df 2: програмування є процес побудови програм, що має
        на меті розв’язання певних проблем. Це визначення
        поєднує вже три поняття пентади.
        \item Df 3: програмування є процес побудови програм, що має
        на меті розв’язання певних проблем, актуальних для
        користувача.
        \item Нарешті, df 4: програмування є процес побудови
        програм, що має на меті розв’язання певних проблем,
        актуальних для користувача, шляхом їх виконання. Це
        визначення поєднує усі поняття нашої пентади.
    \end{enumerate}

    Кожному наведеному визначенню відповідають певні
    методи програмування.
    Так, перше визначення фактично концентрується на
    методах написання синтаксично правильних програм.
    Аналогія - школа.

    Друге визначення вже апелює до проблем предметної
    області, тому відповідні методи програмування мають
    враховувати семантику програм. Продовжуючи
    аналогію – на цьому рівні діти повинні правильно
    викладати думки.

    Трете визначення апелює до користувачів та їх
    особливостей. На цьому рівні дітей у школі повинні
    навчати специфіці звертання та розмов з учителями,
    учнями, людьми похилого віку тощо.

    Нарешті, четверте визначення говорить про те, що
    методи програмування повинні враховувати
    особливості інтерфейсу та обчислюваності програм.
    Таким чином, введена пентада дозволяє скласти певне
    уявлення про програмування. Та все ж уведені поняття
    є ще дуже абстрактними. Побудована пентада фактично
    відображає будь-який процес програмування, зокрема,
    процес підготовки програм для ЗМІ і т.д. Це пентада
    програмування взагалі. Наприклад, за цією пентадою
    ми можемо складати план роботи на наступний
    тиждень.

    Тому необхідно рухатися до більш конкретних понять,
    що розкриють процес програмування саме в
    професійному аспекті, у аспекті інформатизації. Як же
    рухатися далі? Аналізуючи побудовану пентаду,
    можемо дійти висновку, що в них мова йде про
    зовнішні властивості програм та програмування.

    \section{Часткова коректність програм}
    Програма $\pi$ вважається (частково)
    коректною, якщо для довільних
    початкових даних, які задовольняють
    умову $\phi$, результат обчислення (якщо
    воно завершується) задовольняє певну
    умову $\psi$ .

    Програму $\pi$ вважають (частково)
    коректною, якщо для довільних вхідних
    даних, що відповідають умові $\phi$,
    результат обчислення (якщо він
    досяжний) задовольняє умові $\psi$ .

    Нехай $\phi, \psi$ — формули логіки предикатів, а $\pi$ —
    імперативна програма.
    Тоді програма $\pi$ називається частково коректною в
    інтерпретації $I$ відносно передумови $\phi$ і післяумови
    $\psi$, якщо триплет $\phi\{\pi\}\psi$ є виконуватим в інтерпретації $I$,
    тобто:
    \begin{equation*}
        I \vDash \phi\{\pi\}\psi
    \end{equation*}
\end{document}
