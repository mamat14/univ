\newpage
\section*{Вступ}
\label{sec:intro}
\addcontentsline{toc}{section}{\nameref{sec:intro}}
\subsubsection*{Оцінка сучасного стану об’єкта розробки.}
Фронтенд пишеться на мовах які перетворюються згодом в Javascript.
Серед популярних мов які схожі Javascript - Typescript, Flow.
Однак існує можливість писати фроненд на функціональних мовах типу
ReasonML(мова схожа на OCaml, але орієнтована на JS розробників),
або його більш проста мова Elm, яка черпає натхнення з ідей Functional Reactive Programming.
Вибір мови буде розглянутий в наступних частинах.
Далі ми наведемо основні метрики вимірювання швидкості веб сайту.
Наразі `Lighthouse report' який використовую пошуковий двигун з назвою Google, вважає головними метриками якості вебсайту з точки зору швидкості такі шість метрик.
Google більше 90\% пошукових запитів в Україні та світі.
Ці метрики засновані за дослідженнях та постійно змінюються.
\begin{enumerate}
    \item FCP(First contentful paint) Швидкість загрузки 1-го контенту
    \item TTI(Time to interactive) Час який потрібен сторінці до того моменту як вона стає повністю інтерактивною
    \item Speed Index Демонструє швидкість з якою загружається наповнення сторінки
    \item TTB(Total blocking time) Сума усіх періодів між FCP та TTI У яких час задачі перевищував 50 мілісекунд.
    \item LCP(Largest contentful paint) Час загрузки найбільшого елементу.
    \item CLS(Cumulative Layout Shift) Вимірює сумарний рух видимих елементів.
\end{enumerate}

\subsubsection*{Актуальність роботи та підстави для її виконання.}
Дослідження демонструють, що чим швидший сайт, тим більше грошей він може заробити(вищий convertion rate) https://wpostats.com/.

Також вочевидь користувачам більше подобається використовувати швидкі сайти https://web.dev/why-speed-matters/.
Робота зроблена з метою опанування і практики розробки сучасних веб-фронтендів та досліження питань швидкості вебсайтів.
Інтернет сьогодні є дуже вагомою часткою суспільства.
Розробка та дослідження ефективних методів розробки дозволить людям отримувати кращий сервіс та не витрачати
час на погано зроблені сайти, які іноді навіть неможливо використовувати.
Інформація з цієї роботи може розповсюджуватися як короткий текст з розробки швидких сайтів.

Споживачі все частіше покладаються на мобільний телефон для доступу до цифрового контенту та послуг,
і якщо ви подивитесь на аналітику вашого сайту, ви, мабуть, побачите те саме, у ваших власних даних.
Споживачі також вибагливіші, ніж будь-коли, і коли вони отримують досвід на вашому сайті,
вони не просто порівнюють вас із вашими конкурентами, вони оцінюють вас за найкращі в своєму класі послуги,
якими вони користуються щодня.
\subsubsection*{Мета й завдання роботи та можливі сфери застосування.}
Метою роботи є дослідження основних метрик які вимірюють швидкість веб-сайтів та методів їх оптимізації.
Виконання цієї роботи дозволить мені в майбутньому створювати швидкі та коректні сайти.
Ця робота також може використовуватися як короткий довідник з розробки швидких сайтів.
Інтернет став глобальним і при цьому в багатьох людей досі старі телефони зі слабкими процесорами,
їм як і багатьом іншим потрібно мати доступ в інтернет, тож дослідження швидкості сайтів може стати корисним є актуальним.
