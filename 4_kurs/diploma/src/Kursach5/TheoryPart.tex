\newpage
\section{Вибір способу розробки}
В процесі вибору способу розробки було зроблено декілька кроків.
Спочатку я спробував не використовувати сторонніх бібліотек.
Цей підхід виявився складним потребуючим глибоких знань у веб розробці Та потребуючим великих зусиль для досягання
як на мене схожого на той що може бути отриманий з використанням Готових рішень.
Цей спосіб підходить для невеликих проектів але вже існує багато фреймворків які спрощують і таку розробку.

Далі я спробував використати фреймворк React.
Разом з розширенням мови Javascript під назвою JSX Він дозволяє розроблювати фронт енд у у знайомий для більшості програмістів спосіб.
Код сторінки може бути оброблений як і будь-яка інша зміна.
Реактор утилізує ідеї функціонального реактивного програмування. Зміни стану лежать повністю на бібліотеці.
Бібліотека надає Ебі ай для зміни стану але безпосередньо зміна стану з робиться бібліотекою.
Бібліотека показала себе зручною у використанні та такою що є такою що підходить для задачі.
Перші спроби розробки за допомогою Реактор були зроблені на мові ReasonML.
Ця мова є функціональною та Вона виявилися не підходящою для проекту бо з моєї точки зору орієнтована
на великі проекти з великим планшетом а для малих проектів рекомендовано використовувати.
Технології які мають багато ресурсів де можна знайти відповіді на запитання та багато написаний
бібліотек це допоможуть зекономити час і гроші при розробці сайту середнього розміру.

Наступним етапом стала адаптація NextJS За рекомендацію 2-а який є професіональним розробником в контенті я почав вивчати його.
Цей фрейморк надає інфраструктуру для Написання швидких сайтів включаючи платформу для хостингу, prebuild сторінок, SSR, та serverless.
Це дозволяє зосередитись на розробці сайту а не на інфраструктурних питаннях.

Для великах та важливих проектів я не рекомендую використовувати в цю технологію бо вона занадто
багато визначає процес розробки і тож не є достатня рибкою для постійно змінюваних умов.

Більшість провідних компаній які розробляють топ 1 фронт енд у світі використовують тракт.
Це пов'язано з тим що бібліотека є де-факто стандартом розробкиШвидких сайтів має багато
бібліотек наприклад отримання швидкості з'єднання об'єму пам'яті швидкості процесором
з'єднання з інтернетом та інших параметрів із браузерів
у простий спосіб не дивлячись на деталі специфікації які описують деталі отримання цих параметрів з пристроїв.К
\section{Опис основних метрик}
\subsection{FCP}
FCP(First contentful paint) Швидкість загрузки Ця метрика вимірює час який потрібен браузеру для того щоб Загрузити 1-й дом
елемент після того як користувач перейшов на сторінку.
Картинки, SVG та непусті canvas елементи вважаються дом контентом будь-що усередині iframe не включається.
Для оптимізації цієї метрики треба у 1-у чергу за гру жати тільки необхідне. Для цього використовується LazyLoad.
Компоненти тобто Javascript код картинки та інші частини цієї частки сайту яка не потрібна буде відкладено на потім.
Зазвичай 1-е має бути загружено скелет сайту Дім є легкою частиною швидкозагружається та надає користувачам необхідне зворотня зворотній зв'язок.
Не впевнений що загружати Найбільший елемент сторінки 1-м є цілі це образа
бо при низькій швидкості з'єднання з інтернет це може зайняти доволі багато часу.
\subsection{Time to interactive}
Time to interactive(TTI)
Ця метрика є важливо тому що деякі сайти оптимізують видимий контент ціною відкладання інших задач на потім.
Через це користувач може мати поганий досвід користування веб-сайту Бо сторінка буде заблокована і коли користувач нічого не зможу зробити.
Метрика вимірює час який потрібен сторінці до того як вона стане повністю інтерактивне. Сторінка вважається повністю інтерактивною коли:
\begin{enumerate}
    \item Сторінка показала увесь корисний контент, що вимірюється FCP.
    \item Усі опрацьовуючи подій зареєстровані для більшості видимих елементів сторінки.
    \item Сторінка відповідає на Інтер акції озера користувача в рамках п'ятидесяти мілісекунд.
\end{enumerate}

Це метрика вимірюється за допомогою порівняння даних з іншими сайтами. Наприклад якщо 99 відсотків сайтів мають значення ікс і ваш сайт має значенняБільше Рівне ікс то значення цього цієї метрики буде становити 99.

А ця методика може бути покращена за допомогою відкладання виконання непотрібного одразу коду на потім.

\subsection{Speed Index}
Індекс швидкості вимірює як швидко контент візуально з'являється на протязі загрузки сторінки. Вимірювальних у 1-у чергу знімаю відео як за гружається сторінка в браузері і враховує Вираховує візуально прогресів між кадрами.
Для стилізації індекс швидкості можна робити все що пришвидшує роботу столі. Наприклад мінімізаціяРоботи головного потоку зменшення виконання скриптів а також перевірка того що текст залишається видимо упродовжЗагрузки веб шрифтів.

\subsection{Total Blocking Time}
Це метрика вимірюється у мілісекунда.Чарльз до 1-го байта вимірює загальний час на протязі якого сторінка заблоковано і не дозволяє і не відповідає користувачу наприклад кліка мишкою натискати на екран або на клавіатуру.Сума вираховується таким чином це сума усіх блокуючих частин усіх довгих та Сок. Усі та скиЯкі виконуються більше п'ятидесяти мілісекунд їду крапку є довгими. Час після 5 вісті мілісекунд є блокуючим блокуючий частини. Наприклад якщо задачі займають 70 мілісекундБлокуюча порція буде дорівнювати 20 секунд мілісекунд.
Найчастіше частини поганої поганого значення цієї метрики пов'язані з виконанням загруз кою пар син Гом загруз кою паркінгу та виконання не потрібноНепотрібного коду. За допомогою Performance панелі ви можете віднайти основні задачі які лягають на основний потік.
Також допомогає покращення власного коду з точки зору алгоритмічної складності.

\subsection{LCP}
Найбільша значуща картинка вимірює час які потрібно найбільше моменту сторінки я для з'являються у видимій частині і Рендер в екран.
Це є апроксимацію часу загрузки основного контенту. Мінусом цієї метрики за моїм досвідом є те що вона може неправильно визначати основний контент сторінки. НаприкладНаприклад коли на мобільному пристрої за гружається 2 картинки 1 з яких основна і займає весь екран а інша займає наприклад 10 відсотків і при цьому вони мають однаковий розмірОбраховувати лайк Хаус може вважати основною картинкою ту яка за гружається ту яка 2-а. Це потребує Змін коду тільки раді швидкості.
Загальному випадку оптимізація цього значення складається з викладання непотрібної роботи адаптивної загрузки картинок та прилоц та й пере загрузка тег.

\subsection{CLS}
Ця метрика Вимірює наскільки сильно елементи гордиться дергається на сторінці.
Наприклад читаючи текст на сторінці або без попередження текст пере мішається переміщується і ви Втрачаєте місце де читал іноді це навіть може призвести до серйозних проблем. Загалом органі сервіси що дергається є неприємними для користувачів.


