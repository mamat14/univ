\newpage
\section{Оптимізація метрик та прикладах}
Через те що в метро вки дуже пов'язані 1 з 1 я наведу приклади приклади як я зміг оптимізувати сторінку.
Усі ці способи допомогою оптимізувати сторінку.
\subsection{Відкладання загрузки непотрібних компонентів}
Цей прийом є дуже ефективним у оптимізації. Наприклад хедер є доволі складною компонентою сайту бо має адаптуватись до розміру та іноді містить у собіБагато логіки.
Після того як було зроблено лінивезагру ска швидкість 3G була покращена.
Фреймворк Next допомагав у цьому. Для того щоб зробити загрузку адаптивною лінивий треба було лише використати тим динамік імпорт.
Кількість необхідного коду час першої загрузки зменшилися на 20 кілобайт.
Для контролю Cumulative Layout shift Рекомендується використовувати компоненту з назвою скелетон.
Вона дозволяє створити відчуття неперервної загрузки сторінки.
Приклади коду.

\subsection{Адаптивні зображення}
Адаптивні зображення є необхідним підкреслив необхідним для швидкої роботи сторінки.
Стандарт html 5 додав можливість робити загрузку картинок адаптивною тобто враховуючи параметри такі як ширина екрану У в пікселях Тобто стандартизовані піксель, Щільність пікселів тобто наприклад для чотирика дисплею треба вочевидьзагружати більш якісну картинку.
Це робиться приблизно таким чином вказується функція з ширина екрану та щільність пікселів У картинку. Фреймворк Next має можливість автоматичної оптимізації та створення картинок відповідного розміру. Він також переводить картинки у формат в епі.
Це покращує метрики такі як LCP, Та інші.
Приклади коду.

\subsection{Відмова від непотрібних бібліотек}
Гарно правило Є підключати тільки найбільш необхідні бібліотеки. Вони мають бути правильно зробленими дозволяти код speaking тобто код має бути розділений про це пізніше. Гарним Гарним правило є підключати лише фреймворк наприклад матеріал ua та бібліотеку для роботи з формами. Також інші бібліотеки підключати лише за необхідностію І якщо можна зробити щось самостійно і більш-менш швидко та якісно робіть це.
Наприклад спочатку footer сайту був взятий з бібліотеки з третьої сторони.
Зміна коду призвела до збільшення швидкості сайту доволі сильно.
Наприклад на моєму сайті є декілька основних бібліотеку вочевидь фреймворки read не Jess матеріал формі Марк дав.

\subsection{Розділення коду(Code splitting)}
Розділення коду є дуже важливою частиною оптимізації сайту.
Webpack Є основною основним інструментом для збірки сайту.
А суть цього підходу У тому що інструмент збірки лише додає у фінальний код тільки те що де-факто використовується.
Для цього використовується Nameт імпорт.
Деякі інструменти такі як Babel Обіцяють автоматизації цього процесу приклад коду.
Я не зміг налаштувати бейсбол для цієї задачі.
Простим способом рекомендується робити іменовані і порти безпосередня.
Це дозволить пришвидшити сайт.

\subsection{Слайдер}
Я додають цю частину бо витратив багато часу на пошуки ефективного способу додавання слайдер.
Мій рецепт буде використовувати keen slider.
Ця бібліотека є 1 з найкращих в інтернеті. Проте вона потребує деяких оптимізації.
Next має можливість пріоритезувати картинки загрузку картинок.
Цей функціонал засновується на Every стикахІ не є ідеальним.
Для того щоби відкласти будь-яку загрузку картинок Слай беру до моменту поки вони будуть необхідні треба Зробити так звану віртуалізації.
Суть цього методу Проста і заключається у тому щоб за гру жати тільки 1 2 картинки наперед замість усіх.
Це є Ще більш вагомим якщо присутні деяка декілька слайдів. Також у випадку коли слайдів дуже багато. Пріоритетзувати загрузку головного зображення є абсолютною необхідністю.
приклад коду
