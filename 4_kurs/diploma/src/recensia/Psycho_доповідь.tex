\documentclass[a4paper,12pt, titlepage]{article}

\usepackage[utf8]{inputenc}
\usepackage[T2A]{fontenc}

\usepackage[english, ukrainian]{babel}

\usepackage{amsmath,amssymb,mathrsfs}
\usepackage{amsthm}
\usepackage[mathscr]{eucal}
%\usepackage[dvips]{graphicx}
\usepackage[pdftex]{graphicx}
\usepackage{epstopdf}
\usepackage{epigraph}

%\usepackage[centerlast,small]{caption2}
\usepackage{enumerate}
\usepackage{verbatim}
\usepackage{layout}
\usepackage{cite}
%%% remove comment delimiter ('%') and specify parameters if required
%\usepackage[dvips]{graphics}

\usepackage{hyperref}
\tolerance=1000
\hypersetup{colorlinks=true}

\pagestyle{empty}

\usepackage{geometry} % Можливість задавати поля
\geometry{top=15mm}
\geometry{bottom=15mm}
\geometry{left=15mm}
\geometry{right=10mm}

\usepackage{extsizes} % Можливість зробити 14-й шрифт(8pt, 9pt, 10pt, 11pt, 12pt, 14pt, 17pt, 20pt.)

\title{\begin{otherlanguage*}{ukrainian}
           Загальні психологічні особливості студентського періоду розвитку особистості.
\end{otherlanguage*}}
\author{\begin{otherlanguage*}{ukrainian} Маматюсупов Дмитрій \end{otherlanguage*}}
\date{\begin{otherlanguage*}{ukrainian} Листопад 2020 \end{otherlanguage*}}

\begin{document}

    \maketitle

    \tableofcontents

    \section{Актуальність вивчення проблеми адаптації студента}
    Для того, щоб навчити студента необхідним йому і суспільству навичкам
    треба, щоб цей студент вчився. Зазвичай навчання не є дуже цікавим процесом,
    більше того, воно є складним та вимогливим, тобто потребує фокусу студента.
    У випадку якщо в студента є внутрішні суперечності він не зможе робити нічого,
    окрім розв'язку цієї суперечності. Процес розвитку і називається адаптацією.

    Таким чином вивчення проблематики адаптації студента до університету потенційно
    може позитивно впливанути на успіхи студена у навчанні.

    \section{Мета есе}
    Вивчення умов потрапляння та існування студентів в навчальному середовищі
    для кращого розуміння процесу адаптації студента до навчаня.

    \section{Основна частина}

    \subsection{Характеристика молодшого студентського віку}
    До молодшого студентського віку відносяться юнаки та дівчата,
    які проживають стадію «юності» (по Еріксону). Ідентичність проти
    дифузії (отроцтво / юність). Це центральна ланка в схемі Еріксона,
    стадія, на якій юнаки та дівчата активно намагаються синтезувати весь
    свій досвід, щоб знайти стійке відчуття особистої ідентичності. Хоча
    цей процес є психосоціальним по природі, що передбачає
    обов'язкове поява соціального відповідності або «солідарності з
    груповими інтересами », Еріксон особливо підкреслює роль самопізнання і
    випробування дійсністю в формуванні ідентичності. Індивідуум
    починає дивитися на себе як на продукт свого життєвого досвіду, гостро
    усвідомлюючи його зв'язність і спадкоємність. Позитивний ров'язок
    попередніх криз, в результаті яких людина стає в міру
    довірливою, самостійною, цілеспрямованою і працьовитою,
    полегшує формування ідентичності, тоді як невдачі в їх подоланні
    можуть призвести до розмивання ідентичності. Вірність, або здатність
    зберігати взяті на себе зобов'язання не дивлячись на несумісні системи
    цінностей, і є та чеснота, яка з'являється в цей період життя.

    Поняття ідентичності акумулює в собі процеси структурної і
    смислової перебудови самосвідомості, які яскраво розгортаються в
    юнацькому віці. У цей період складається не тільки відносно-стійке
    смислове ядро ​​особистості, але і формуються психологічні
    інструменти подальшого самовизначення. Однак саме сьогодні
    людина на шляху до дорослості особливо вразлива в плані пошуку варіанту позитивної
    ідентичності, що пов'язано з особливостями сучасної соціокультурної
    ситуації в Україні. Підліток, у якого тільки кристалізується образ
    світу, особливо чутливий до криз соціального устрою суспільства,
    основне загрозливе прояв яких - дифузія і розмивання
    цінностей.

    Розвиток людини в період юності може йти декількома шляхами.
    Юність може бути бурхливою: пошуки сенсу життя, свого місця в цьому світі
    можуть стати особливо напруженими. Деякі старшокласники плавно і
    безперервно просуваються до переломного моменту життя, а потім
    відносно легко включаються в нову систему відносин. Вони більше
    цікавляться загальноприйнятими цінностями, в більшій мірі
    орієнтуються на оцінку оточуючих, авторитет дорослих. Можливі й
    різкі, стрибкоподібні зміни, які завдяки добре розвиненою
    саморегуляції не викликають складнощів у розвитку.

    При переході від підліткового віку до юнацького відбувається
    зміна у ставленні до майбутнього: якщо підліток дивиться на майбутнє з
    позиції сьогодення, то юнак дивиться на сьогодення з позиції майбутнього.
    Вибір професії та типу навчального закладу неминуче диференціює
    життєві шляхи юнаків і дівчат, закладає основу їх соціально-психологічних
    і індивідуально-психологічних відмінностей. Навчальна
    діяльність стає навчально-професійною, реалізує
    професійні та особистісні устремління юнаків і дівчат. Провідне
    місце у старшокласників займають мотиви, пов'язані з самовизначенням
    і підготовкою до самостійного життя, з подальшим утворенням і
    самоосвітою. Ці мотиви набувають особистісний сенс і
    стають значущими [3, 5].

    \subsection{Формування ідентичності}
    Джеймс Марсіа (James Marcia, 1966, 1980, 1993) розвинув теорію
    Еріксона і визначив чотири різних стани, або види формування,
    ідентичності. До видів, або «статусів ідентичності», відносяться:
    безумовний вид, дифузія, мораторій і досягнення ідентичності. При
    цьому враховується, чи пройшов індивідуум через період прийняття рішення,
    званий \textit{кризою ідентичності}, і прийняв він зобов'язання по
    відношенню до певного ряду виборів, таких як система цінностей або
    план майбутнього професійного шляху.

    Підлітки, що знаходяться в \textit{статусі зумовленості}, взяли на себе
    зобов'язання, не пройшовши через процес прийняття рішення. Вони вибрали
    професію, релігійні погляди, ідеологію та інші аспекти своєї
    ідентичності. Однак цей вибір був зроблений ними раніше і визначено швидше
    їх батьками або вчителями, ніж ними самими. Їх перехід до дорослості
    проходить рівно і стикається лише з незначними конфліктами, але в
    ньому немає активного експериментування.

    Молодь, у якій відсутнє почуття напрямку, а можливо, і
    немає вираженого бажання знайти його, знаходиться в \textit{дифузному статусі}.
    Такі підлітки не пройшли через кризу і не вибрали собі
    професійну роль або моральний кодекс. Вони просто уникають
    зіткнення з цією проблемою. Деякі з них зосереджені на
    моментальне задоволенні своїх потреб і бажань; інші
    експериментують, не маючи конкретних планів і цілей, з різними
    установками і видами поведінки (Cote, Levine, 1988).
    Підлітки або молоді дорослі в \textit{статусі мораторію} знаходяться в
    середині триваючої кризи ідентичності або періоду прийняття
    рішень. Ці рішення можуть стосуватися вибору професії, релігійних чи
    етичних цінностей, політичної філософії. Молодь в цьому статусі
    поглинена «пошуком самих себе».

    І, нарешті, \textit{досягнення ідентичності} є статусом,
    досягається тими, хто пройшов через кризу ідентичності і прийняв свої
    зобов'язання. У підсумку вони самі вибрали собі роботу і намагаються жити по
    самостійно сформульованому моральному кодексу. Досягнення
    ідентичності зазвичай розглядається як найбільш бажаний і зрілий
    статус (Marcia, 1980).

    \subsubsection{Гендерні відмінності.}
    Марсіа і інші дослідники звернули увагу на суттєву різницю між чоловіками і жінками в
    поведінці і установках, пов'язаних з різними статусами ідентичності.
    Наприклад, чоловіки в статусах досягнення ідентичності і мораторію
    мають, як правило, високу самооцінку. У жінок ці періоди
    характеризуються наявністю великої кількості невирішених конфліктів,
    що відносяться до сім'ї і вибору кар'єри.

    Подальші дослідження частково підтвердили початкові
    дані, але дозволили отримати більш глибоке уявлення про проблему.
    Наприклад, Саллі Арчер (Saly Archer, 1985) виявила, що по відношенню до
    сім'ї та вибору кар'єри дівчинки-старшокласниці частіше знаходяться в статусі
    вирішеним, а хлопчики - в стані дифузії. Більш того, дівчатка в
    статусах безумовності і мораторію проявляли велику
    невизначеність при залагодженні конфліктів, пов'язаних з кар'єрними та
    сімейними вподобаннями. Однак як хлопчики, так і дівчатка говорять,
    що вони планують одружитися, ростити дітей і робити кар'єру. Дівчатка
    зазвичай частіше висловлювали своє занепокоєння щодо можливих
    конфліктів між сім'єю і кар'єрою. При питанні про ступінь занепокоєння,
    випробовується респондентами, 75\% чоловіків і 16\% жінок повністю
    заперечували його наявність, 25\% чоловіків і 42\% жінок визнали, що в
    деякій мірі відчувають його, в той час як 0\% чоловіків і 42\% жінок
    заявили, що відчувають сильне занепокоєння щодо можливих
    конфліктів між сім'єю і кар'єрою. Було також виявлено, що хоча у
    чоловіків розвивається перш за все внутрішня ідентичність, у
    жінок формується поєднання як внутрішньо-особистісної, так і
    міжособистісної ідентичності (Lytel, Bakken & Roming, 1997) [2].

    \textit{Розвиток студента на різних курсах має деякі особливі риси.}
    \begin{enumerate}
        \item \textbf{Перший курс} вирішує завдання залучення недавнього абітурієнта до
        студентським формам колективного життя. Поведінка студентів відрізняється
        високим ступенем конформізму; у першокурсників відсутній
        диференційований підхід до своїх ролей.
        \item \textbf{Другий курс} - період найбільш напруженою навчальної діяльності
        студентів. У житті другокурсників інтенсивно включені всі форми
        навчання і виховання. Студенти отримують загальну підготовку,
        формуються їх широкі культурні запити і потреби. процес
        адаптації до даного середовища в основному завершено.
    \end{enumerate}
    В цьому відношенні дуже цікавий аналіз студентської молоді в
    зв'язку з обраною ними професією. Вся сукупність сучасних
    студентів досить явно розділяється на три групи.
    Першу складають студенти, орієнтовані на освіту як на
    професію. У цій групі найбільше число студентів, для яких інтерес
    до майбутньої роботи, бажання реалізувати себе в ній - найголовніше. Лише вони
    відзначили схильність продовжувати свою освіту в аспірантурі. Усе
    інші фактори для них менш значущі. У цій групі близько третини студентів.

    Другу групу складають студенти, орієнтовані на бізнес. Вона
    становить близько 26\% від загального числа опитаних. Відношення до
    освіти у них зовсім інше: для них освіта виступає в якості
    інструменту для того, щоб у подальшому спробувати створити власну справу, зайнятися торгівлею і т.п.
    Вони розуміють, що з часом і ця сфера потребує освіти, але до
    своєї професії вони відносяться менш зацікавлено, ніж перша група.

    Третю групу складають студенти, яких, з одного боку, можна
    назвати "не визначилися", з іншого - задавленими різними
    проблемами особистого, побутового плану. На перший план у них виходять
    побутові, особисті, житлові, сімейні проблеми. Можна було б сказати,
    що це група тих, хто "пливе за течією" - вони не можуть вибрати свого
    шляху, для них освіту і професія не представляють того інтересу, як у
    перших груп. Можливо, самовизначення студентів даної групи
    відбудеться пізніше, але можна припустити, що в цю групу потрапили люди,
    для яких процес самовизначення, вибору шляху, цілеспрямованості характерний.

    Привчені до щоденної опіки і контролю в школі, деякі
    першокурсники не вміють приймати елементарні рішення. У них
    недостатньо виховані навички самоосвіти і самовиховання.

    Відомо, що методи навчання у вузі різко відрізняються від шкільних,
    так як в середній школі навчальний процес побудований так, що він весь час
    спонукає учня до занять, змушує його працювати регулярно, інакше
    дуже швидко з'явиться маса двійок. В іншій обстановці потрапляє вчорашній
    школяр, переступивши поріг вузу: лекції, лекції, лекції. Коли ж
    починаються семінари, до них теж виявляється можна не завжди готуватися.
    Загалом, не треба кожен день щось вчити, вирішувати, запам'ятовувати. В
    Внаслідок нерідко виникає думка про уявній легкості навчання в вузі
    в першому семестрі, формується впевненість можливості все надолужити і
    освоїти перед сесією, виникає безтурботне ставлення до навчання. Розплата
    настає на сесії [4].
    \subsection{Адаптація}
    Багато першокурсники на перших порах навчання зазнають
    великі труднощі, пов'язані з відсутністю навичок самостійної
    навчальної роботи, вони не вміють конспектувати лекції, працювати з
    підручниками, знаходити і добувати знання з першоджерел, аналізувати
    інформацію великого обсягу, чітко і ясно викладати свої думки.
    Адаптація студентів до навчального процесу (за даними вивчення
    регуляторної функції психіки) закінчується в кінці 2-го - початку 3-го
    навчального семестру.

    Однією з головних причин, що ускладнюють адаптацію до умов
    навчання в інституті, понад 50\% опитаних назвали брак часу
    для самостійної роботи при підготовці домашніх завдань. У зв'язку з
    цим майже 25\% студентів приходять на заняття непідготовленими.

    Однією з основних завдань роботи з першокурсниками є
    Розробка та впровадження методів раціоналізації й оптимізації
    самостійної роботи.

    Процес адаптації кожного студента йде по-своєму. Юнаки та
    дівчата, які мають трудовий стаж, легше і швидше адаптуються до умов
    студентського життя і побуту.

    У проведених дослідженнях процесу адаптації першокурсників до
    вузу зазвичай виділяються наступні головні труднощі: негативні
    переживання, пов'язані з доглядом вчорашніх учнів зі шкільного
    колективу з його взаємною допомогою і моральною підтримкою;
    невизначеність мотивації вибору професії, недостатня
    психологічна підготовка до неї; невміння здійснювати психологічне
    саморегулювання поведінки і діяльності, які згубно відсутністю
    звички до повсякденного контролю педагогів; пошук оптимального
    режиму праці та відпочинку в нових умовах; налагодження побуту і
    самообслуговування, особливо при переході з домашніх умов в
    гуртожиток; нарешті, відсутність навичок самостійної роботи, невміння
    конспектувати, працювати з першоджерелами, словниками, довідниками,
    покажчиками.

    Всі ці труднощі різні за своїм походженням. Одні з них
    об'єктивно неминучі, інші носять суб'єктивний характер і пов'язані зі
    слабкою підготовкою, дефектами виховання в сім'ї та школі.

    \textit{Соціальна адаптація студентів у вузі ділиться на:}
    \begin{enumerate}
        \item професійну адаптацію, під якою розуміється
        пристосування до характеру, змісту, умов і організації навчального
        процесу, вироблення навичок самостійності в навчальній і науковій
        роботі;
        \item соціально-психологічну адаптацію - пристосування індивіда
        до групи, взаєминам з якою, вироблення власного стилю
        поведінки.
    \end{enumerate}
    Інакше кажучи, «під адаптаційної здатністю розуміється
    здатність людини пристосовуватися до різних вимог середовища
    (Як соціальним, так і фізичним) без відчуття внутрішнього
    дискомфорту і без конфлікту з середовищем ».

    \textit{Адаптація} - це передумова активної діяльності і необхідна
    умова її ефективності. У цьому позитивне значення адаптації для
    успішного функціонування індивіда в тій чи іншій соціальної ролі.
    Дослідники розрізняють три форми адаптації студентів-першокурсників
    до умов вузу:
    \begin{enumerate}
        \item адаптація формальна, що стосується пізнавально
        інформаційного пристосування студентів до нового оточення, до
        структурі вищої школи, до змісту навчання в ній, її вимогам, до
        своїх обов'язків;
        \item громадська адаптація, тобто процес внутрішньої інтеграції
        (Об'єднання) груп студентів-першокурсників і інтеграція цих же груп
        зі студентським оточенням в цілому;
        \item дидактична адаптація, що стосується підготовки студентів до
        нових форм і методів навчальної роботи у вищій школі [4].
        типи студентів.
    \end{enumerate}

    \subsubsection{Класифікація}
    У роботі хотілося б зупинитися на типах студентів,
    що виділяються за різними критеріями. Наведу поділ з причин
    вступу до ВНЗ і по відношенню до навчання.

    Виділені типи більш детальні, ніж
    простий розподіл на "бюджетних" і "комерційних", причому ці типи
    зустрічаються і в тій і в іншій з описаних груп.

    Перший тип умовно можна назвати "підприємцем". Цей студент
    воліє досягнення успіху в сфері бізнесу, здобуває вищу
    освіту для того, щоб освоїти теорію і практику
    підприємництва, швидко просуватися по службі, займатися
    керівної, організаторською діяльністю, він упевнений в правильності
    вибору спеціальності, навчання, відповідно йому своїх здібностей, але в
    Водночас він більш критичний до свого навчального закладу, краще знає
    специфіку професії (можливості професійного зростання, розмір
    заробітної плати, умови праці, перспективи службової кар'єри), які не
    побоюється безробіття, у нього більш розвинені (по самооцінці) такі
    актуальні особистісні якості, як індивідуалізм, професіоналізм,
    підприємливість, самостійність, здатність змінювати погляди при
    зміні обставин, швидка адаптація і легка вживаемости в нові
    умови.

    Другий тип з тією ж часткою умовності називають "емігрант". Вища
    освіту "емігранти" в більшій мірі отримують для того, щоб
    вільно оволодіти іноземними мовами, отримати можливість навчання,
    роботи за кордоном. Вони впевнені в правильності власного вибору
    спеціальності і відповідно йому своїх здібностей, а також в
    можливості вузу дати їм підготовку на необхідному рівні. У них добре
    розвинені (по самооцінці) індивідуалізм, життєвий оптимізм, легка
    вживаемости в нові умови.

    Обом цим типам протистоїть "традиціоналіст". Він цінує хорошу
    освіту, професійну підготовку, отримує вищу освіту
    для того, щоб отримати диплом, вести наукові дослідження, менш
    критичний по відношенню до вузу, гірше знає реалії подальшої
    професійної діяльності, більше побоюється безробіття, у нього
    сильно розвинені професіоналізм і працездатність, менш -
    підприємливість, здатність ризикувати, змінювати погляди при зміні
    обставин, вживатися в нові умови, життєвий оптимізм.

    Існує також ще один поділ студентів на групи. За
    відношенням до навчання ряд дослідників виділяють п'ять груп.

    До першої групи належать студенти, які прагнуть опанувати
    знаннями, методами самостійної роботи, набути професійних
    вміння і навички, шукають способи раціоналізації навчальної діяльності.
    Навчальна діяльність для них - необхідний шлях до хорошого оволодіння
    обраною професією. Вони відмінно навчаються з усіх предметів навчального
    циклу. Інтереси цих студентів зачіпають широке коло знань, більш
    широкий, ніж передбачено програмою. Вони активні в усіх сферах
    навчальної діяльності. Студенти цієї групи самі активно шукають аргументи,
    додаткові обґрунтування, порівнюють, зіставляють, знаходять істину,
    активно обмінюються думками з товаришами, перевіряють достовірність
    своїх знань.

    До другої групи належать студенти, які прагнуть придбати
    знання у всіх сферах навчальної діяльності. Для цієї групи характерно
    захоплення багатьма видами діяльності, але їм швидко набридає глибоко
    вникати в суть тих чи інших 12* предметів і навчальних дисциплін. Ось
    чому вони нерідко обмежуються поверхневими знаннями. Основний
    принцип їх діяльності - найкраще потроху. Вони не витрачають
    багато зусиль на конкретні справи. Як правило, добре вчаться, але отримують
    часом незадовільні оцінки з предметів, які їх не
    цікавлять.

    До третьої групи відносяться студенти, які виявляють інтерес
    тільки до своєї професії. Придбання знань і вся їх діяльність
    обмежуються вузькопрофесійними рамками. Для цієї групи
    студентів характерно цілеспрямоване, вибіркове придбання
    знань, і тільки необхідних (на їхню думку) для майбутньої
    професійної діяльності. Вони багато читають додаткової
    літератури, глибоко вивчають спеціальну літературу, ці студенти добре
    і відмінно навчаються з предметів, пов'язаних зі своєю спеціальністю; у той же
    час не проявляють належного інтересу до суміжних наук і дисциплін
    навчального плану.

    До четвертої групи належать студенти, які непогано вчаться, але до
    навчальній програмі відносяться вибірково, проявляють інтерес тільки до тих
    предметів, які їм подобаються. Вони несистематично відвідують навчальні
    заняття, часто пропускають лекції, семінарські і практичні заняття, які не
    проявляють інтересу до певних видів навчальної діяльності та
    дисциплін навчального плану, так як їх професійні інтереси ще не
    сформовані.

    До п'ятої групи належать ледарі. У вуз вони прийшли по
    наполяганням батьків або "за компанію" з товаришем, або для того, щоб
    не йти працювати і не потрапити в армію. До навчання ставляться байдуже,
    постійно пропускають заняття, мають "хвости", їм допомагають товариші, і
    найчастіше вони дотягують до диплома [1].

    \section{Висновок}
    \epigraph{Вчитись, вчитись, вчитись.}{Стіна факультету кібернетики}
    В університеті треба вчитись. Адаптація є невід'ємною частиною для початку плодотворної праці.
    Аналіз показав те, що студенство не підготовлене до цього. Ми побачили, що студенти діляться
    на велику кількість груп за різними класифікаціями, кожна з яких має свою специфіку адаптації.
    Суперечності в студентському індівіді мають різний характер, але потребують вирішення. Це
    спостереження в одному із своїх проявів дає нам етапи формування особистості.
    Зауважу, що безумовний етап є на мою думку домінантним в Україні, враховуючи конформізм,
    нормативність університету та армії. Я бажаю , щоб менше студентів задавали собі питання
    "Що я тут забув?" по відношенню до університету і сподіваюсь, що в майбутньому все більше з них
    буде йти до університету виходячи із власних роздумів. В цьому випадку адаптація буде простішою і
    буде мати лише технічний характер, що дозволить студентам все ж таки робити, те, навіщо все це
    створювалось - вчитись.

    \begin{thebibliography}{9}
        \bibitem{bulanova-toporikova}
        Буланова-Топоркова М. В.
        \textit{Педагогіка і психологія вищої школи: Навчальний посібник.}
        Ростов н/Д: Фенікс, 2002.

        \bibitem{craig}
        Грейс Крайг, Дон Бокум
        \textit{Психологія розвитку}
        СПб : Питер, 2005.

        \bibitem{personality_theory}
        \textit{Теорії особистості}

        \bibitem{age_psych}
        \textit{Вікова психологія}

        \bibitem{ericson}
        Еріксон Е.
        \textit{Ідентичність: юність і криза}
        «Прогрес», 1996.
    \end{thebibliography}
\end{document}
