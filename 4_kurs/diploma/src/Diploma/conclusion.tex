\newpage
\section*{Висновки}
\label{sec:conclusion}
\addcontentsline{toc}{section}{\nameref{sec:conclusion}}
    Робота присвячена досліженню структури деяких важливих класів $C^*$ - алгебр.
    та є складовою частиною програми Еліота класифікації простих ядерних $C^*$ - алгебр.

    Основним результатом був опис $K_0$ та $K_1$ груп деформації Ріфеля алгебр Кунца
    та алгебри функцій на одновимірному торі.

    В роботі були основні результати К-теорії, такі,
    як властивості функтора $\widetilde{K}_0$, функтора $\widetilde{K}_1$
    такі як слабка точність, гомотопічна інваріантність, неперервність та стабільність.
    Також був досліджений зв'язок між цими функторами (теорема Бота).
    Були обраховані деякі К-групи.
    За допомогою вивченого я обрахував К-групу для $C^*$ - алгебри $C^1(\mathbf{T})$.
    Також мною була обрахована К-групи тензорного добутку $\mathcal{O}_n \otimes_* \mathcal{O}_m$,
    разом з цим автоматично їх деформації Ріфеля.

    Після написання роботи стало зрозумілим актуальність тематики та
    об'єкта дослідження.
    Степінь застованості отриманих результатів оцінюю як помірну, потребуючу
    подальших досліджень.
    Кількість застосування К-теорії дуже \\ велика.
    Одним з найважливіших результатів є теорема Ат'ї-Зінгера про індекс, \\
    дослідження якої може стати наступним кроком у вивченні предмету.

