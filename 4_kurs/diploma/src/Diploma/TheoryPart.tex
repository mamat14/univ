\newpage
\section{$C^*$-алгебри}\label{sec:theoretical-statements}
Інформація цього параграфа взята з $[8]$.
Нехай A - алгебра.
Інволюцією алгебри називається операція * така, що:
\begin{align}
    a^*^* &= a, \\
    (ab)^* &= b^* a^*
\end{align}

Множину $(A,*)$ будемо називати *-алгерою.
Перетин довільної кількості *-алгебр знову є *-алгеброю.
З цього випливає, що для довільної множини S існує найменша *-алгебра яка її містить.

Для множини $S$ з $A$, визначимо $S^* = \{s^*| s \in S\}$.
Якщо $S = S^*$, то S називається самоспряженою множиною.
Фактор-алгебра $A/I$ є *-алгеброю, якщо $I$ - самоспряжений ідеал.
Інволюція на цій алгебрі визначається за правилом:$(a+I)^*=a^*+I$

Елемент $a \in A$ називається самоспряженим, якщо $a^* = a$.
Для кожного $a \in A$ існує єдине його зображення у вигляді $a = b + ic$,
де b,c -- самоспряжені елементи, $b = \frac{(a + a^*)}{2}, c = \frac{a - a^*}{2i}$
Множину усіх самоспряжених елементів будемо позначати $A_{sa}$.

Елемент називається нормальним, якщо $a^*a = aa^*$.
Породжена таким \\
елементом *-алгебра є комутативною та співпадає з
лінійною оболонкою \\ елементів $a^m a^*^n$, де $m,n \in N, m+n > 0$.
\begin{example}
    Нехай a - оператор правого зсуву на $R^2$,
    тоді a не є нормальним оператором.

    Нескладно зрозуміти, що $a^*$ - оператор лівого зсуву.
    Тоді
    \begin{align*}
        a^* a[(x,y)] &= a^*[(0,x)] = (x,0), \\
        a a^*[(x,y)] &= a[(y,0)] = (0,y)
    \end{align*}
\end{example}

Елемент називається проектором, якщо  $p = p^* = p^2$.

Якщо алгебра є унітальною, то $1^* = 1 1^* = (1 1^*)^* = 1$.
Якщо $a \in Inv(A)$, то $(a^*)^{-1} = (a^{-1})^*$.
Отже $\sigma(a^*) = \{\bar{\lambda} | \lambda \in \sigma(a) \}$

Елемент називається унітарним, якщо $u^* u = u u^* = 1$.
Якщо $u^* u = 1$, то елемент називається ізометрією, якщо $u u^* = 1$, то коізометрією.

*-гомоморфізмом *-алгебр будемо називати гомоморфізм такий, що $\phi(a^*) = \phi(a)^*$.
*-ізоморфізмом називається бієктивний *-гомомофізм. \\
Якщо $\phi$ - *-гомоморфізм, то $\ker(\phi)$ - самоспряжений ідеал в A, а $\phi(A)$ - є *-алгеброю.

*-автоморфізмом називється *-ізоморфізм *-алгебри на себе. \\
Якщо A-унітальна, то для довільного унітарного елемента u відображення
\[
    Ad u: a \rightarrow u^* a u
\]
є *-автоморфізмом A\@.

Елементи $a,b$ називаються унітарно еквівалентними, якщо $b = u^* a u$, для якогось унітарного $u$.
Унітарна еквівалентність є відношенням еквівалентності на $A$.
Унітарно еквівалентні елементи мають однаковий спектр.

Банаховою *-алгеброю називається *-алгебра, у якій виконоється рівність
\begin{equation}
    || a^* || = || a ||
\end{equation}
Банахова *-алгебра називається унітальною, якщо вона містить одиницю, таку, що, $||1|| = 1$.

$C^*$-алгеброю називається банахова *-алгебра у якій виконується рівність
\begin{equation}
    ||a^* a || = || a ||^2
\end{equation}

Якщо C^*-алгебри містять одиницю, то $||1|| = 1$.
Аналогічно, якщо p - \\
ненульовий проектор, то $||p|| = 1$.
Якщо u- унітарний елемент - то $||u||^2 =||u^* u|| = ||1|| = 1$.
Далі, $\lambda \in \sigma(u) \implies (\lambda)^{-1} \in \sigma(u^{-1}) = \sigma(u^*)$.
З того, що $||\lambda|| < ||u|| = 1$, та $||\lambda^{-1}|| < ||u^*|| = 1$ випливає, що $||\lambda|| = 1$.

\begin{theorem}
    Якщо a-самоспряжений елемент, то $r(a) = ||a||$.
    \begin{proof}
        \begin{equation*}
            ||a^2|| = ||a||^2 \implies ||a^{2^n}|| = ||a||^{2^n} \implies \Lim_{n \rightarrow \infty} ||a^n||^{\frac{1}{n}} = \Lim_{n \rightarrow \infty} ||a^{2^n}||^{\frac{1}{2^n}} = ||a||
        \end{equation*}
    \end{proof}
\end{theorem}

\begin{corollary}
    На *-алгебрі існує не більше однієї норми, яка робить з неї $C^*$ - алгебру.
    \begin{proof}
        \begin{equation*}
            ||a||^2 = ||a^* a|| = r(a^* a),
        \end{equation*}
        для довільної норми.
    \end{proof}
\end{corollary}

\begin{lemma}
    Якщо на банаховій алгебрі існує інволюція, така, що $||a a^*|| \geq ||a||^2$, то ця алгебра є $C^*$ - алгеброю.
    \begin{proof}
        \begin{align*}
            ||a||^2 \leq ||a a^*||, ||a a^*|| \leq ||a|| ||a^*|| \implies \\
            \forall a ||a|| \leq ||a^*|| \implies ||a|| = ||a^*|| \implies ||a||^2 = ||a a^*||
        \end{align*}
    \end{proof}
\end{lemma}

Довільну алгебру A можно зробити унітальною.
Для цього нехай $\widetilde{A} = A \oplus \mathbb{C}$, як векторні простори.
Визначимо множення в A за правилом: $(a, \lambda) (b, \beta) = (ab + \lambda b + \beta a, \lambda \beta)$.
Відображення $\phi: \widetilde{A} \rightarrow \mathbb{C}, \phi: (a, \lambda) \rightarrow \lambda$, називається канонічним
гомоморфізмом.
Ця алгебра називається уніталізацією A.
Одиницею у ній є елемент $(0,1)$.
Алгебру A можна вкласти в $\widetilde{A}$ за допомогою \\
ін'єктивного гомоморфізма $\phi: a \rightarrow (a,0)$.
Цей гомомофізм є унітальм, його ядро дорівнює A .
Якщо A - комутативна алгебра -- то такою буде і $\widetilde{A}$.
Якщо A - нормована, то такою буде і $\widetilde{A}$.
\begin{equation*}
    ||a + \lambda|| = ||a|| + |\lambda|
\end{equation*}
%\begin{example}
%    Let A -- C*-algebra. $\dim A < \infty$. Then...
%\end{example}
%\begin{example}
%    Let A -- C*-algebra. A -- is commutative...
%\end{example}
%\begin{example}
%    Let A -- C*-algebra generated by isometrics.
%\end{example}
