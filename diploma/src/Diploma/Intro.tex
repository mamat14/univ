\newpage
\section*{Вступ}
\label{sec:intro}
\addcontentsline{toc}{section}{\nameref{sec:intro}}
\subsubsection*{Оцінка сучасного стану об’єкта розробки.}
К-теорія є дуже широкою дисципліною яка пов'язана з алгебраїчною \\
топологією, алгебраїчною геометрією, теорією операторів та з багатьма іншими
розділами математики.
В моїй роботі вивчається К-теорія $C^*$ - алгебр.
Серед досягнень К-теорії доведення теореми про індекс Ат'ї-Зінгера, яка
сприяла \\ знаходженню нових зв'язків між топологією та математичною фізикою \\
за доведення якої автори отримали премію Філдса.
Також К теорія є важливим інструментом класифікації $C^*$ - алгебр.
Однією з головних ідей є співставлення $C^*$ - алгебрі деякої групи,
вивчаючи властивості якої, можна зробити висновки про саму $C^*$ - алгебру.
Розвиток К-теорія отримала в роботах Гельфанда, \\ Наймарка, Кунца, Еліота.
Узагалінення К-теорії, так звана КК-теорія, була запропонована
Каспаровим.
$C^*$-алгебри використовуються у теоретичній фізиці.
\subsubsection*{Актуальність роботи та підстави для її виконання.}
К-теорія на даний момент є дисципліною, яка активно розвивається.
Вона дуже широко застосовується та дає методи для роз'язання задач,
які раніше здавалися дуже складними.
Тож вивчення К-теорії може стати гарним  \\ інструментом, для подальших досліджень.
Найбільше робота пов'язана з [1], [2], [9].
\subsubsection*{Мета й завдання роботи та можливі сфери застосування..}
Метою роботи є дослідження $C^*$ - алгебр та К-теорії, навчання навичкам
роботи з літературою, розширення кругозору.
Як говорилось вище К-теорія має велику кількість застосувань,
серед яких топологія, математична фізика та інші.
