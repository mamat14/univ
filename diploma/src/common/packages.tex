%%% Кодировки и шрифты %%%
\usepackage[utf8]{inputenc}
% Украинский язык
%\usepackage[english]{babel}
\usepackage[ukrainian]{babel}

%%%% Установки для размера шрифта 14 pt %%%%
\usepackage[14pt]{extsizes}
%библиография в содержании
\usepackage[nottoc]{tocbibind}
%%% Математические пакеты %%%
\usepackage{amsthm,amsmath,amscd}   % Математические дополнения от AMS
\usepackage{amsfonts,amssymb}       % Математические дополнения от AMS
\usepackage{mathtools}              % Добавляет окружение multlined
\usepackage{stmaryrd }
\usepackage{tikz}
\usepackage{xcolor}
\usetikzlibrary{cd, babel}

\renewcommand\qedsymbol{$\blacksquare$}
\DeclareMathOperator{\Ima}{Im}

\theoremstyle{plain}
\newtheorem{theorem}{Теорема}[section]
\newtheorem{lemma}{Лема}[section]
\newtheorem{proposition}{Пропозиція}[section]
\newtheorem{axiom}{Аксіома}[section]

% %        Подчиненные счетчики в окружениях
% %\newtheorem{lemmac}[theorem]{Лемма}
% % Теорема 1, предложение 2, гипотеза 3...

% %        Двойная нумерация окружений
% \newtheorem{theorems}{Теорема}[section]
% \newtheorem{lemmas}{Лемма}[section]
% % Теорема 1.1, лемма 1.1, предложение 1.1 и т.п.

%        Ненумерованные окружения
\newtheorem*{theoremnn}{Теорема}
% Просто теорема (без номера)
% Окружения с прямым шрифтом
\theoremstyle{Визначення}
\newtheorem{definition}{Означення}[section]
\newtheorem{corollary}{Наслідок}[section]
\newtheorem{remark}{Зауваження}[section]
\newtheorem{example}{Приклад}[section]

\usepackage{geometry}
 \geometry{
 a4paper,
 left=25mm,
 top=20mm,
 right=10mm,
 bottom=20mm
 }                            % Для последующего задания полей
\usepackage{breqn}
%%% Оформление абзацев %%%
\usepackage{indentfirst}                            % Красная строка
%%% Гиперссылки %%%
\usepackage{hyperref}
%%% Изображения %%%
\usepackage{graphicx}                  % Подключаем пакет работы с графикой
\newcommand{\Lim}[1]{\raisebox{0.5ex}{\scalebox{0.8}{$\displaystyle \lim_{#1}\;$}}}
