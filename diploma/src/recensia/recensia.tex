\documentclass[a4paper,12pt]{report}

\usepackage[utf8]{inputenc}
\usepackage[T2A]{fontenc}

\usepackage[english, ukrainian]{babel}

\usepackage{amsmath,amssymb,mathrsfs}

\usepackage{amsthm}
\usepackage[mathscr]{eucal}
%\usepackage[dvips]{graphicx}
\usepackage[pdftex]{graphicx}
\usepackage{epstopdf}


%\usepackage[centerlast,small]{caption2}
\usepackage{enumerate}
\usepackage{verbatim}
\usepackage{layout}
\usepackage{cite}
%%% remove comment delimiter ('%') and specify parameters if required
%\usepackage[dvips]{graphics}

\usepackage{hyperref}
\tolerance=1000
\hypersetup{colorlinks=true}

\pagestyle{empty}

\usepackage{geometry} % Можливість задавати поля
\geometry{top=15mm}
\geometry{bottom=15mm}
\geometry{left=15mm}
\geometry{right=10mm}

\usepackage{extsizes} % Можливість зробити 14-й шрифт(8pt, 9pt, 10pt, 11pt, 12pt, 14pt, 17pt, 20pt.)



\begin{document}


%%% remove comment delimiter ('%') and select language if required
%\selectlanguage{spanish}

    \begin{center}
        \textbf{РЕЦЕНЗІЯ \\}
        \textit{на випускну кваліфікаційну роботу \\
            на здобуття ступеня бакалавра \\
            студента 4 курсу денної форми навчання \\
            факультету комп'ютерних наук та кібернетики \\
            Маматюсупова Дмитрія Олеговича \\
            на тему \\ ,,K-теорія $C^*$-алгебр''}

    \end{center}

    \vspace{3mm}

    Рецензована випускна кваліфікаційна робота на здобуття ступеня бакалавра присвячена вивченню К-теорії $C^*$ - алгебр.
    Проблематика є актуальною, бо К-теорія є важливим інструментом класифікації С*-алгебр та має застосування
    в топології, наприклад в теорії вузлів.
    Робота складається зі реферату, вступу, трьох розділів, висновків і списку літератури.
    Розділ 1 містить базову інформацію про $C^*$ - алгебри.
    В розділі 2 розглядаються основні теореми та визначення К-теорії $C^*$ - алгебр, вивчаються властивості функторів
    $K_0$, $K_1$ такі як слабка точність, гомотопічна інваріантність, неперервність та стабільність. Також
    вивчається зв'язок між цими функторами висвітелний в теоремі Бота.
    Розділ 3 дає короткий огляд на деформацію Ріфеля, а потім приводить обрахування груп $K_0, K_1$ для
    $C(\mathbf{T}), \mathcal{O}_i \otimes \mathcal{O}_j$ та автоматично для їх деформіцій Ріфеля.

    Робота написана грамотною науковою українською мовою.
    Автор продемонстрував, що володіє сучасним математичним апаратом і вміє застосовувати методи вивчаємої теорії.
    Вважаю, що випускна кваліфікаційна робота бакалавра студента 4 курсу Маматюсупова Дмитрія Олеговича на
    тему ,,К-теорія $C^*$ - алгебр'' виконана на належному рівні,
    задовольняє всім вимогам до випускних кваліфікаційних робіт бакалавра і заслуговує оцінки ,,відмінно'' (95 балів).

    \vspace{3mm}


%%%%%%%%%%%%%%%%%%%%%%%%%%%%%%%%%%%%%%%%%%%%%%%%%%%%%%%%%%%%%%%%555555
    \noindent
    Доцент кафедри МСС\\
    Київського національного університету\\
    імені Тараса Шевченка \\
    доктор фізико-математичних наук \hspace{4cm} В.~В.~Пічкур




    \pagestyle{empty}


\end{document}
