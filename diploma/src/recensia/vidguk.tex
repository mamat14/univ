\documentclass[a4paper,12pt]{report}

\usepackage[utf8]{inputenc}
\usepackage[T2A]{fontenc}

\usepackage[english, ukrainian]{babel}

\usepackage{amsmath,amssymb,mathrsfs}

\usepackage{amsthm}
\usepackage[mathscr]{eucal}
%\usepackage[dvips]{graphicx}
\usepackage[pdftex]{graphicx}
\usepackage{epstopdf}


%\usepackage[centerlast,small]{caption2}
\usepackage{enumerate}
\usepackage{verbatim}
\usepackage{layout}
\usepackage{cite}
%%% remove comment delimiter ('%') and specify parameters if required
%\usepackage[dvips]{graphics}

\usepackage{hyperref}
\tolerance=1000
\hypersetup{colorlinks=true}

\pagestyle{empty}

\usepackage{geometry} % Можливість задавати поля
\geometry{top=15mm}
\geometry{bottom=15mm}
\geometry{left=15mm}
\geometry{right=10mm}

\usepackage{extsizes} % Можливість зробити 14-й шрифт(8pt, 9pt, 10pt, 11pt, 12pt, 14pt, 17pt, 20pt.)



\begin{document}


%%% remove comment delimiter ('%') and select language if required
%\selectlanguage{spanish}

    \begin{center}
        \textbf{Відгук \\}
        \textit{на випускну кваліфікаційну роботу \\
            на здобуття ступеня бакалавра \\
            студента 4 курсу денної форми навчання \\
            факультету комп'ютерних наук та кібернетики \\
            Маматюсупова Дмитрія Олеговича \\
            на тему \\ ,,K-теорія $C^*$-алгебр''}

    \end{center}

    \vspace{3mm}
    Робота Маматюсупова Д.О. присвячена досліженню структури деяких важливих класів $C^*$ - алгебр.
    Робота є актульною та є складовою частиною програми Еліота класифікації простих ядерних $C^*$ - алгебр.

    Основним результатом був опис $K_0$ та $K_1$ груп деформації Ріфеля алгебр Кунца та алгебри функцій на одновимірному торі.
    Для отримання цих результатів знадобилось опанувати багато розділів сучасної теорії операторів та операторних алгебр
    і можна сказати що він впорався з цією частиною роботи бездоганно.

    Всю роботу над дипломом Маматюсупов Д.О. виконав абсолютно самостійно,
    науковий керівник приймав участь у постановці початкової задачі та консультував.
    Вважаю що Маматюсупов Д.О. має всі дані для того щоб при бажанні стати перспективним молодим науковцем і його кваліфікаційна
    робота без сумніву заслуговує на найвищу оцінку.

    \vspace{3mm}


%%%%%%%%%%%%%%%%%%%%%%%%%%%%%%%%%%%%%%%%%%%%%%%%%%%%%%%%%%%%%%%%555555
    \noindent
    Доцент кафедри ДО\\
    Київського національного університету\\
    імені Тараса Шевченка \\
    доктор фізико-математичних наук \hspace{4cm} Д.~П.~Проскурін




    \pagestyle{empty}


\end{document}
